%タイトル
\date{\today}


%パッケージ
\usepackage{amsmath, amssymb, amsfonts, physics, mathrsfs}		%数式
%\usepackage{emath, emathFx}										%数式拡張(未実装)
\usepackage{physics}
\usepackage{siunitx}													%単位
\usepackage{fancyhdr, lastpage}										%ヘッダー&フッター
\usepackage{hyperref}													%リンク
\usepackage{url}														%URL
\usepackage{pxjahyper}												%リンク修正
\usepackage{overcite}													%引用
\usepackage{pxrubrica}												%ルビ
\usepackage[table]{xcolor}											%表&色
\usepackage{longtable, float, multirow, array, listliketab, enumitem, tabularx}
\usepackage{tcolorbox}												%tcolorbox
\usepackage{listings, plistings}										%ソースコードの埋め込み
\usepackage{graphicx}													%図
\usepackage{subcaption, wrapfig}									%図の配置
\usepackage{tikz}														%TikZ
\usetikzlibrary{calc, patterns, decorations, angles, calendar, backgrounds, shadows, mindmap}
\usepackage{import, grffile}											%ファイル管理
\usepackage{standalone}
\usepackage{xparse}													%マクロ作成用
\usepackage[T1]{fontenc}											%フォント
\usepackage{textcomp}
\usepackage[utf8]{inputenc}
\usepackage{lmodern}
\usepackage{mathptmx}
\usepackage[scaled]{helvet}
\renewcommand{\ttdefault}{pcr}
\usepackage{bm}


%hyperrefのsetup
\hypersetup{%
	bookmarks = true,%
	bookmarksnumbered = true,%
	hidelinks,%
	colorlinks = true,%
	linkcolor = black,%
	urlcolor = cyan,%
	citecolor = black,%
	filecolor = magenta,%
	setpagesize = false,%
	pdftitle = {},%
	pdfauthor = {R.Morita},%
	pdfkeywords = {},%
	}


%siunitxのsetup
\sisetup{%
	%detect-family = true,%						%太字の設定
	detect-inline-family = math,%
	detect-weight = true,%
	detect-inline-weight = math,%
	%input-product = *,%							%*で乗算
	quotient-mode = fraction,%					%/で分数に
	fraction-function = \frac,%
	inter-unit-product = \ensuremath{\hspace{-1.5pt}\cdot\hspace{-1.5pt}},%	%単位の区切りを\cdotに
	per-mode = symbol,%							%単位の区切り(分数)を/に
	product-units = single,%						%複数の数値にわたって単位を付けるとき最後にだけ付ける
	}


%字間
\kanjiskip 0pt plus 0pt minus 0pt
\xkanjiskip 0.25zw plus 0.125zw minus 0.125zw
%行間
\setlength{\lineskiplimit}{6pt}
\setlength{\lineskip}{6pt}
%数式インデント
\setlength{\mathindent}{5zw}


%引用符を肩付けに
\renewcommand{\citeform}[1]{[#1]}


%文字式の単位〔roman〕
\newcommand{\unitis}[1]{%	
	\mathrm{〔#1〕}
	}


%マクロ
%丸(EmathFxにpzdオプションを付けると有効)
%\let\maru\pzdmaru


